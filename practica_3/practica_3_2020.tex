\documentclass[11pt,a4paper,twoside]{article}%
\usepackage{amsmath}
\usepackage{amsfonts}
\usepackage{amssymb}
\usepackage{graphicx}%
\usepackage[latin1]{inputenc}
\usepackage[spanish]{babel}

\usepackage{color}

\setcounter{MaxMatrixCols}{30}
%TCIDATA{OutputFilter=latex2.dll}
%TCIDATA{Version=5.50.0.2953}
%TCIDATA{CSTFile=40 LaTeX article.cst}
%TCIDATA{Created=Thursday, December 26, 2013 11:29:10}
%TCIDATA{LastRevised=Friday, March 28, 2014 13:58:37}
%TCIDATA{<META NAME="GraphicsSave" CONTENT="32">}
%TCIDATA{<META NAME="SaveForMode" CONTENT="1">}
%TCIDATA{BibliographyScheme=Manual}
%TCIDATA{<META NAME="DocumentShell" CONTENT="Standard LaTeX\Blank - Standard LaTeX Article">}
%BeginMSIPreambleData
\providecommand{\U}[1]{\protect\rule{.1in}{.1in}}
%EndMSIPreambleData
\newtheorem{theorem}{Theorem}
\newtheorem{acknowledgement}[theorem]{Acknowledgement}
\newtheorem{algorithm}[theorem]{Algorithm}
\newtheorem{axiom}[theorem]{Axiom}
\newtheorem{case}[theorem]{Case}
\newtheorem{claim}[theorem]{Claim}
\newtheorem{conclusion}[theorem]{Conclusion}
\newtheorem{condition}[theorem]{Condition}
\newtheorem{conjecture}[theorem]{Conjecture}
\newtheorem{corollary}[theorem]{Corollary}
\newtheorem{criterion}[theorem]{Criterion}
\newtheorem{definition}[theorem]{Definition}
\newtheorem{example}[theorem]{Example}
\newtheorem{exercise}[theorem]{Exercise}
\newtheorem{lemma}[theorem]{Lemma}
\newtheorem{notation}[theorem]{Notation}
\newtheorem{problem}[theorem]{Problem}
\newtheorem{proposition}[theorem]{Proposition}
\newtheorem{remark}[theorem]{Remark}
\newtheorem{solution}[theorem]{Solution}
\newtheorem{summary}[theorem]{Summary}
\newenvironment{proof}[1][Proof]{\noindent\textbf{#1.} }{\ \rule{0.5em}{0.5em}}
\topmargin -0.5in
\oddsidemargin -0.1in
\evensidemargin -0.1in
\textwidth 6.5in
\textheight 9.5in
\begin{document}

\begin{center}
\textbf{\textsf{Estad\'{\i}stica (Qu\'{\i}mica) - Primer Cuatrimestre- 2020 - Coronavirus}}\\
\vspace{0.2cm}
\textbf{Pr\'{a}ctica 3 - Sumas de variables aleatorias\vspace{-0.1in}}
\end{center}

\begin{enumerate}
\item Se realizan mediciones independientes del volumen inicial $\left(
X\right)  $ y final $\left(  Y\right)  $ en una bureta. Supongamos que las
mediciones inicial y final siguen el modelo de errores independientes, es
decir,
\[
X=\alpha+\varepsilon_{X},\qquad Y=\beta+\varepsilon_{Y}%
\]
donde $\alpha$ y $\beta$ son los vol\'{u}menes desconocidos, $\varepsilon_{X}$
y $\varepsilon_{Y}$, los errores de medici\'{o}n, son variables aleatorias
independientes con media 0 y varianza $\sigma^{2}$. Sea $Z=Y-X$.

\begin{enumerate}
\item Hallar $\mathbb{E}(Z)$ y $\mathbb{V}(Z)$.

\item En una titulaci\'{o}n, la lectura inicial en una bureta es de 3.51ml y
la lectura final es de 15.67ml. Para ambas mediciones se sabe que el error de
medici\'{o}n tiene una desviaci\'{o}n est\'{a}ndar de 0.02ml.

\begin{enumerate}
\item Calcular el valor estimado del volumen utilizado.

\item \textquestiondown Cu\'{a}l es la desviaci\'{o}n est\'{a}ndar de su error
de medici\'{o}n?
\end{enumerate}
\end{enumerate}




\item \label{medidas repetidas} Modelo para medici\'on  con error aditivo.  Se 
desea determinar una magnitud $\mu.$ Para ello se realizar\'an $n$ medidas 
repetidas, es decir, se realizar\'{a}n $n$ mediciones de la misma magnitud en 
id\'{e}nticas condiciones, que denotaremos con $X_{1},\dots,X_{n}$. Asumimos 
el siguiente modelo para las variables aleatorias $X_{i}$
\[
X_{i}=\mu+\varepsilon_{i}
\]
donde $\mu$ es la verdadera magnitud desconocida, y $\varepsilon_{i}$ es la
variable aleatoria que denota el error de la i\'{e}sima medici\'{o}n. Asumimos
que $\varepsilon_{1},\dots,\varepsilon_{n}$ son variables aleatorias
independientes e id\'{e}nticamente distribuidas (v.a.i.i.d.) con esperanza
cero y varianza $\sigma^{2}=0.25$. Notar que los errores son no observables. El
supuesto de que $\mathbb{E}(\varepsilon_{i})=0$ refleja la creencia en que el
m\'{e}todo de medici\'{o}n empleado es exacto. Es decir que no produce errores
sistem\'{a}ticos. La varianza del error, 
$\sigma^{2}=\mathbb{V}(\varepsilon_{i})$
representa la precisi\'{o}n del m\'{e}todo de medici\'{o}n empleado. Sea
\[
\overline{X}_{n}=\frac{1}{n}\sum_{i=1}^{n}X_{i}
\]
el promedio (o media muestral) de las $n$ observaciones. 


Asumir ahora que el error de medici\'on tiene una distribuci\'on normal con media cero. Este modelo
probabil\'istico se conoce como el Modelo de Gauss sin sesgo. Asumir tambi\'en que el error de medici\'{o}n tiene desv\'io est\'andar $\sigma=0.5$, o sea que $\varepsilon_{i}\sim\mathcal{N}(0;0.25)$.
\begin{enumerate}
	\item  Obtener la distribuci\'{o}n de $\overline{X}_{n}$, su esperanza y
	su varianza.
	
	\item Calcular la probabilidad de que el promedio de $n=10$ mediciones y
	de $n=100$ mediciones diste de la verdadera magnitud $\mu$ en menos de 0.1
	unidades. Notar que no fue necesario conocer el valor de $\mu$ para realizar
	este c\'{a}lculo. 
	
	\item Obtener una expresi\'{o}n para la probabilidad de que el promedio de 
$n$ mediciones diste de la verdadera magnitud $\mu$ en menos de $0.1$ 
unidades en funci\'{o}n de $n$. Estudiar la  monoton\'{\i}a y el l\'{\i}mite 
cuando $n$ tiende a infinito de esta probabilidad.
	
	\item  Determinar cu\'an grande debe ser $n$ para que $P\left(  |\overline
	{X}_{n}-\mu|<0.1\right)  \geq0.99$. 
	
	
{\color{red}\item Para incluir en el futuro, en alguna parte de esta practica. Calcular la siguiente probabilidad
$$\mathbb P\left(\overline X_n-\frac{1.96*\sqrt{0.25}}{\sqrt{n}} <\mu <\overline X_n+\frac{1.96*\sqrt{0.25}}{\sqrt{n}}\right)$$}	
\end{enumerate}

\item \label{medidas repetidas_bis} Considerar nuevamente el modelo de 
mediciones propuesto en el ejercicio anterior, suponiendo ahora que 
$\varepsilon_i$  tiene una distribuci\'{o}n \textbf{desconocida}, pero se sabe 
que 
$$E(\varepsilon_i)=0\;, \quad V(\varepsilon_i)=0.25$$


\begin{enumerate}
	\item Hallar $\mathbb{E}(\overline{X}_{n})$ y 
$\mathbb{V}(\overline{X}_{n})$.
	
	\item Para $n=10$ y $n=100$ mediciones, usando la desigualdad de Chebyshev,
	encontrar una cota inferior para
	\[
	P\left(  |\overline{X}_{n}-\mu|<0.1\right)  .
	\]
	Comparar el resultado obtenido con el hallado en el \'{\i}tem (b) del 
ejercicio anterior.
	
	\item Determinar cu\'an grande debe ser $n$ para que $P\left(  |\overline{X}%
	_{n}-\mu|<0.1\right)  \geq0.99$, usando nuevamente la desigualdad de Chebyshev.
		Comparar el resultado obtenido con el valor hallado en el \'{\i}tem 
(d) del ejercicio anterior.
	
\end{enumerate}


\item Se desea conocer la proporci\'{o}n de personas que est\'{a}n a favor de
la despenalizaci\'{o}n del aborto en una ciudad. Sea $p$ la proporci\'{o}n
poblacional que est\'{a} a favor de la despenalizaci\'{o}n. Observar que $p$ es
un n\'{u}mero fijo y desconocido. Para estimar a $p$ se eligen $n$ personas al
azar y se les pregunta a cada una de ellas su opini\'{o}n. Para $i$ entre $1$
y $n$, sean%
\[
X_{i}=\left\{
\begin{array}
[c]{lll}%
1 &  & \text{si la i\'{e}sima persona encuestada est\'{a} a favor de la
despenalizaci\'{o}n}\\
&  & \\
0 &  & \text{en otro caso.}%
\end{array}
\right.
\]
Asumimos que las $X_{i}$ son v.a.i.i.d.

\begin{enumerate}
\item Expresar la proporci\'{o}n muestral de encuestados a favor de la
despenalizaci\'{o}n en t\'{e}rminos de las variables $X_{i}.$

\item Proponer un estimador para $p$ a partir de las variables $X_{i}.$
Observar que el estimador es una variable aleatoria.

\item Hallar una cota superior para la probabilidad de que el estimador y el
verdadero par\'{a}metro $p$ difieran en m\'{a}s de 0.1, que no dependa de
valores desconocidos. Es decir, acotar superiormente la siguiente
expresi\'{o}n,
\[
P\left(  \left\vert \overline{X}_{n}-p\right\vert >0.1\right)
\]
de modo que la cota s\'{o}lo dependa de $n$. \textquestiondown Qu\'{e} pasa
con esta probabilidad cuando $n$ aumenta? \textquestiondown C\'{o}mo puede el
encuestador mejorar su estimaci\'{o}n de $p$?
\end{enumerate}
%
%\item \label{medidas repetidas}Se desea determinar una magnitud $\mu.$ Para
%ello se realizan $n$ medidas repetidas, es decir, se realizar\'{a}n $n$
%mediciones de la misma magnitud en id\'{e}nticas condiciones, que denotaremos
%con $X_{1},...,X_{n}$. Asumimos el siguiente modelo para las variables
%aleatorias $X_{i}$%
%\[
%X_{i}=\mu+\varepsilon_{i}%
%\]
%donde $\mu$ es la verdadera magnitud desconocida, y $\varepsilon_{i}$ es la
%variable aleatoria que denota el error de la i\'{e}sima medici\'{o}n. Asumimos
%que $\varepsilon_{1},...,\varepsilon_{n}$ son variables aleatorias
%independientes e id\'{e}nticamente distribuidas (v.a.i.i.d.) con esperanza
%cero y varianza $\sigma^{2}=0.25$. Note que los errores son no observables. El
%supuesto de que $E(\varepsilon_{i})=0$ refleja la creencia en que el
%m\'{e}todo de medici\'{o}n empleado es exacto. Es decir no produce errores
%sistem\'{a}ticos. La varianza del error, $\sigma^{2}=Var(\varepsilon_{i})$
%representa la precisi\'{o}n del m\'{e}todo de medici\'{o}n empleado. Sea
%\[
%\overline{X}_{n}=\frac{1}{n}\sum_{i=1}^{n}X_{i}%
%\]
%el promedio (o media muestral) de las $n$ observaciones.
%
%\begin{enumerate}
%\item Halle $E(\overline{X}_{n})$ y $Var(\overline{X}_{n})$.
%
%\item Para $n=10$ y $n=100$ mediciones, usando la desigualdad de Chebyshev,
%encuentre una cota inferior para
%\[
%P\left(  |\overline{X}_{n}-\mu|<0.1\right)  .
%\]
%
%
%\item Determine cuan grande debe ser $n$ para que $P\left(  |\overline{X}%
%_{n}-\mu|<0.1\right)  \geq0.99$, usando nuevamente la desigualdad de Chebyshev.
%\end{enumerate}
%
%Asuma ahora que el error de medici\'{o}n tiene una distribuci\'{o}n normal con
%media cero. Este modelo probabil\'{\i}stico se conoce como el Modelo de Gauss
%sin sesgo. Asuma tambi\'{e}n que la desviaci\'{o}n est\'{a}ndar es $0.5$, es
%decir $\varepsilon_{i}\sim\mathcal{N}(0,0.25)$.
%
%\begin{enumerate}
%\item[(d)] Obtenga la distribuci\'{o}n de $\overline{X}_{n}$, su esperanza y
%su varianza.
%
%\item[(e)] Calcule la probabilidad de que el promedio de $n=10$ mediciones y
%de $n=100$ mediciones diste de la verdadera magnitud $\mu$ en menos de 0.1
%unidades. Note que no fue necesario conocer el valor de $\mu$ para realizar
%este c\'{a}lculo. Compare los resultados obtenido con los valores hallados en
%el \'{\i}tem b).
%
%\item[(f)] Obtenga una expresi\'{o}n para la probabilidad de que la
%medici\'{o}n diste de la verdadera magnitud $\mu$ en menos de $0.1$ unidades
%en funci\'{o}n de $n$. Estudie su monoton\'{\i}a y el l\'{\i}mite cuando $n$
%tiende a infinito de esta probabilidad.
%
%\item[(g)] Determine cuan grande debe ser $n$ para que $P\left(  |\overline
%{X}_{n}-\mu|<0.1\right)  \geq0.99$. Compare el resultado obtenido con el valor
%hallado en el \'{\i}tem c).
%\end{enumerate}

\item Considerar nuevamente el modelo propuesto en el ejercicio
\ref{medidas repetidas}, asumiendo ahora que los errores se 
distribuyen de manera uniforme: $\varepsilon_{i}\sim
\mathcal{U}(-\sqrt{3}/2,\sqrt{3}/2)$. Calcular de forma aproximada la
probabilidad $P\left(  |\overline{X}_{n}-\mu|<0.1\right)  $, para $n=100$ y
determinar cu\'an grande debe ser $n$ para que $P\left(  |\overline{X}_{n}%
-\mu|<0.1\right)  \geq0.99$. Comparar con los resultados obtenidos en  el 
ejercicio \ref{medidas repetidas} y en el ejercicio \ref{medidas repetidas_bis}. 
\textquestiondown Qu\'e observa?

%\item La figura siguiente muestra
%\[%
%%TCIMACRO{\FRAME{itbpFX}{15.6531cm}{8.1693cm}{0cm}{}{}%
%%{graficobinomialpracticacontinuas.jpg}{\special{ language "Scientific Word";
%%type "GRAPHIC";  maintain-aspect-ratio TRUE;  display "USEDEF";
%%valid_file "F";  width 15.6531cm;  height 8.1693cm;  depth 0cm;
%%original-width 7.6666in;  original-height 3.9894in;  cropleft "0";
%%croptop "0.9989";  cropright "1.0003";  cropbottom "0";
%%filename 'graficobinomialpracticacontinuas.JPG';file-properties "XNPEU";}}}%
%%BeginExpansion
%\raisebox{-0cm}{\fbox{\includegraphics[
%trim=0.000000in 0.000000in -0.002300in 0.004388in,
%natheight=3.989400in,
%natwidth=7.666600in,
%height=8.1693cm,
%width=15.6531cm
%]%
%{graficobinomialpracticacontinuas.jpg}%
%}}%
%%EndExpansion
%\]
%la funci\'{o}n de probabilidad puntual en $k$ graficada sobre el intervalo de
%longitud uno centrado en $k,$ para todos los enteros $k$ entre 0 y 100, junto
%con una curva normal $\mathcal{N}(\mu,\sigma^{2}).$ La funci\'{o}n reci\'{e}n
%descripta est\'{a} definida por
%\[
%g\left(  x\right)  =\left\{
%\begin{array}
%[c]{lll}%
%p_{X}\left(  k\right)  &  & \text{si }k-\frac{1}{2}\leq x<k+\frac{1}{2},\text{
%para }k\text{ entero entre 0 y 100}\\
%&  & \\
%0 &  & \text{en otro caso}%
%\end{array}
%\right.
%\]
%donde $X\sim Bi\left(  100,0.5\right)  .$\allowbreak
%
%\begin{enumerate}
%\item \textquestiondown Cu\'{a}les son los valores de $\mu$ y $\sigma^{2}$ que
%corresponden a la curva normal dibujada?
%
%\item La probabilidad de obtener 52 caras en 100 tiradas de una moneda
%equilibrada, \textquestiondown es exactamente igual al \'{a}rea entre $51.5$ y
%$52.5$ bajo la curva normal o al \'{a}rea bajo el gr\'{a}fico de $g$?
%\textquestiondown Son muy distintas entre s\'{\i}? \textquestiondown Qu\'{e}
%resultado te\'{o}rico est\'{a} usando?
%
%\item La probabilidad de obtener 52 caras o m\'{a}s en 100 tiradas de una
%moneda equilibrada puede ser aproximada por \textquestiondown cu\'{a}l zona
%del gr\'{a}fico? Calc\'{u}lela.
%\end{enumerate}

\item Se tira 100 veces un dado de 6 caras. Usar la aproximaci\'{o}n normal
para hallar la probabilidad de que:

\begin{enumerate}
\item salga \textquotedblleft6\textquotedblright\ entre 15 y 20 veces, inclusive.

\item la suma de los resultados obtenidos sea menor que 300.

\item el n\'{u}mero de veces que el resultado sea par est\'{e} entre 40 y 60
veces, inclusive.

\item el n\'{u}mero de veces que el resultado sea par sea mayor o igual que el
n\'{u}mero de veces que el resultado sea impar.
\end{enumerate}

\item Un negocio de mascotas ofrece el siguiente servicio para sus clientes
que toman vacaciones. El servicio consiste en alquilarle al due\~{n}o de la
mascota un dispenser autom\'{a}tico de raciones diarias de alimento
balanceado. Cuando el animal termina de comer una raci\'{o}n,
autom\'{a}ticamente, el dispenser pone a su disposici\'{o}n la raci\'{o}n
siguiente. El tiempo (en d\'{\i}as) que el gato de Felipe demora en comer una
raci\'{o}n de alimento es una variable aleatoria Exponencial con par\'{a}metro
$\lambda=2$. Se puede suponer que los tiempos que tarda en comer cada
raci\'{o}n son independientes entre s\'{\i}.

\begin{enumerate}
\item Felipe se va 30 d\'{\i}as de vacaciones y contrata este servicio con 62
raciones. Si el gato no tiene comida disponible se escapa de la casa.
\textquestiondown Cu\'{a}l es, aproximadamente, la probabilidad de que Felipe
encuentre al gato en su hogar al volver de las vacaciones?

\item \textquestiondown Cu\'{a}ntas raciones tiene que comprar si quiere
encontrar al gato en su casa al volver de las vacaciones con una probabilidad aproximada
mayor o igual que $0.99$?
\end{enumerate}

%\item Comparar lo obtenido en el ejercicio anterior con los resultados que se obtienen sin aproximar utilizando R. \textquestiondown Es buena la aproximaci\'on?

\item Para rellenar una zona del r\'{\i}o se utilizan 2 camiones (A y B). La
distribuci\'{o}n de la carga diaria (en toneladas m\'{e}tricas) transportada
por el cami\'{o}n A tiene funci\'{o}n de densidad
\[
f\left(  x\right)  =\left\{
\begin{array}
[c]{lll}%
\frac{x}{52} &  & \text{si }11<x<15,\\
&  & \\
0 &  & \text{en otro caso}%
\end{array}
\right.
\]
El cami\'{o}n B lleva una carga diaria en toneladas con esperanza $18$ Tm y
desv\'{\i}o est\'{a}ndar $1.3$ Tm.

\begin{enumerate}
\item Calcular esperanza y varianza de la carga diaria transportada por A.

\item Calcular esperanza y desv\'{\i}o est\'{a}ndar de la carga total llevada
por los dos camiones en un d\'{\i}a, asumiendo que las cargas transportadas
por ambos camiones son independientes.

\item Calcular \textbf{aproximadamente} la probabilidad de que la carga total
transportada en 256 d\'{\i}as est\'{e} entre 7950 y 8000 Tm.  Asuma que las
cargas transportadas en d\'{\i}as distintos son independientes.

\item \textquestiondown Puede calcular la probabilidad pedida en c)
\textbf{exactamente}?
\end{enumerate}

\item Supongamos que el peso (en gramos) de un frasco de mermelada sigue una
distribuci\'{o}n normal con un peso medio de $\mu_{A}$ para la marca A y
$\mu_{B}$ para la marca B, y con desv\'{\i}os $\sigma_{A}=8g$ y $\sigma
_{B}=6g$, respectivamente. Se eligen al azar $n_{A}=40$ frascos de la marca A
y $n_{B}=35$ frascos de la marca B, y se pesan sus contenidos.

\begin{enumerate}
\item Defina las variables aleatorias involucradas en este problema. Indique
la distribuci\'{o}n exacta o aproximada de las medias muestrales asociadas a
ambas marcas. Indique la distribuci\'{o}n exacta o aproximada de la diferencia
entre la media muestral para la marca A y la media muestral para la marca B.
Enuncie los resultados te\'{o}ricos en los que se basa. Observe que no es
necesario conocer los valores de $\mu_{A}$\ y $\mu_{B}$\ sino s\'{o}lo de la
diferencia para poder determinar la distribuci\'{o}n de $\overline{X}_{n_{A}%
}-\overline{Y}_{n_{B}}.$

\item Asumamos que los pesos medios o esperados de ambas marcas son iguales.
Calcular la probabilidad de que la distancia entre las medias muestrales sea a
lo sumo 3.

\item Calcular la probabilidad de que la distancia anterior sea por lo menos 5,
asumiendo que los pesos medios de ambas marcas son iguales.

\item No se conocen los valores de $\mu_{A}$\ y $\mu_{B}$, s\'{o}lo se sabe
que el valor observado de la diferencia entre la \textbf{media muestral} para
la marca A y la \textbf{media muestral} para la marca B es 5,
\textquestiondown ser\'{\i}a razonable inferir que $\mu_{A}-\mu_{B}=0$? Para
contestar calcule, asumiendo que $\mu_{A}-\mu_{B}=0$, la probabilidad de que
la distancia entre las medias muestrales sea 5 \'{o} m\'{a}s. Si esta
probabilidad fuera \textquotedblleft muy peque\~{n}a\textquotedblright\ uno
tender\'{\i}a a dudar de que la diferencia entre las medias poblacionales es 0.

\item \textquestiondown Se modifican las distribuciones de a) si $n_{A}=400$ y
$n_{B}=350$? Justificar.
\end{enumerate}
\end{enumerate}


\end{document}
