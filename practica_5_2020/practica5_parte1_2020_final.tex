\documentclass[11pt,a4paper,twoside]{article}%
\usepackage{amsmath}
\usepackage{amsfonts}
\usepackage{amssymb}
\usepackage{graphicx}%
\setcounter{MaxMatrixCols}{30}
%TCIDATA{OutputFilter=latex2.dll}
%TCIDATA{Version=5.50.0.2953}
%TCIDATA{CSTFile=40 LaTeX article.cst}
%TCIDATA{Created=Thursday, December 26, 2013 11:29:10}
%TCIDATA{LastRevised=Thursday, February 13, 2014 10:53:32}
%TCIDATA{<META NAME="GraphicsSave" CONTENT="32">}
%TCIDATA{<META NAME="SaveForMode" CONTENT="1">}
%TCIDATA{BibliographyScheme=Manual}
%TCIDATA{<META NAME="DocumentShell" CONTENT="Standard LaTeX\Blank - Standard LaTeX Article">}
%BeginMSIPreambleData
\providecommand{\U}[1]{\protect\rule{.1in}{.1in}}
%EndMSIPreambleData
\newtheorem{theorem}{Theorem}
\newtheorem{acknowledgement}[theorem]{Acknowledgement}
\newtheorem{algorithm}[theorem]{Algorithm}
\newtheorem{axiom}[theorem]{Axiom}
\newtheorem{case}[theorem]{Case}
\newtheorem{claim}[theorem]{Claim}
\newtheorem{conclusion}[theorem]{Conclusion}
\newtheorem{condition}[theorem]{Condition}
\newtheorem{conjecture}[theorem]{Conjecture}
\newtheorem{corollary}[theorem]{Corollary}
\newtheorem{criterion}[theorem]{Criterion}
\newtheorem{definition}[theorem]{Definition}
\newtheorem{example}[theorem]{Example}
\newtheorem{exercise}[theorem]{Exercise}
\newtheorem{lemma}[theorem]{Lemma}
\newtheorem{notation}[theorem]{Notation}
\newtheorem{problem}[theorem]{Problem}
\newtheorem{proposition}[theorem]{Proposition}
\newtheorem{remark}[theorem]{Remark}
\newtheorem{solution}[theorem]{Solution}
\newtheorem{summary}[theorem]{Summary}
\newenvironment{proof}[1][Proof]{\noindent\textbf{#1.} }{\ \rule{0.5em}{0.5em}}
\topmargin -0.2in
\oddsidemargin -0.2in
\evensidemargin -0.2in
\textwidth 7in
\textheight 9in
\begin{document}

\begin{center}
\textbf{\textsf{Estad\'{\i}stica (Qu\'{\i}mica) - Primer Cuatrimestre - 2020 - Coronavirus}}\\

\textbf{Pr\'{a}ctica 5 - Estimaci\'{o}n - Intervalos de Confianza - Primera parte\vspace
{-0.1in}}
\bigskip

\end{center}

En todos los ejercicios propuestos

\begin{enumerate}
\item[a)] defina las variables aleatorias y los par\'{a}metros involucrados.

\item[b)] de ser posible indique:

\begin{enumerate}
\item[i.] la distribuci\'{o}n de las variables aleatorias

\item[ii.] el significado intuitivo de los par\'{a}metros.
\end{enumerate}

\item[c)] compare los resultados de hacer las cuentas a mano con las salidas
obtenidas con el R, de manera de verificar las primeras y aprender a usar las
segundas, en aquellos ejercicios en los que ambas cosas sean posibles.
\end{enumerate}

\bigskip

{\bf Ejercicios}

\begin{enumerate}
\item Sea $X_{1},\ldots.X_{n}$ una muestra aleatoria de una distribuci\'{o}n
con media $\mu$ y varianza $\sigma^{2}$.

\begin{enumerate}
\item Pruebe que $\overline{X}_{n}^{2}$ no es un estimador insesgado de
$\mu^{2}$.

\item \textquestiondown Para qu\'{e} valores de $k$ es $\widehat{\mu}%
^{2}=\left(  \overline{X}_{n}^{2}-ks^{2}\right)  $ un estimador insesgado de
$\mu^{2}$?
\end{enumerate}

\item Sea $X_{1},\ldots.X_{n}$ una muestra aleatoria de una distribuci\'{o}n
Bernoulli de par\'{a}metro $p$ y sea $T_{n}=%
%TCIMACRO{\dsum \limits_{i=1}^{n}}%
%BeginExpansion
{\displaystyle\sum\limits_{i=1}^{n}}
%EndExpansion
X_{i}$. Consideremos un nuevo par\'{a}metro $\theta=p\left(  1-p\right)  $.
Muestre que
\[
\frac{T_{n}\left(  n-T_{n}\right)  }{n\left(  n-1\right)  }%
\]
\ es un estimador insesgado de $\theta$.

\item Para determinar la constante de un resorte, se puede medir la oscilaci\'on de una masa $m$ fijada en uno de sus extremos. Mediante
la f\'ormula $k= 4\pi^2m/T^2$ donde $k$ es la constante del resorte y $T$ es el per\'iodo de oscilaci\'on de dicha masa. 
 Se realizan mediciones exactas de la masa y del per\'iodo. Asumamos que 
cada medici\'on de la masa y del per\'iodo siguen distribuciones  normales, con varianza 0.1 y 0.05, respectivamente. 
De $5$ mediciones de la masa y $4$ del per\'iodo se obtuvieron medias muestrales $0.52$ y $1.3$. 
Asumiendo que todas las mediciones son independientes,
estime la constante del resorte e indique el valor del error de estimaci\'on asociado a la estimaci\'on. 



\item A partir de un gran n\'{u}mero de mediciones, se sabe que un m\'{e}todo
para determinar la cantidad de manganeso en un mineral comete errores
aleatorios con distribuci\'{o}n normal de media cero y desviaci\'{o}n
est\'{a}ndar 0.09.

\begin{enumerate}
	\item Se hicieron 5 mediciones de un mismo mineral y se obtuvo un valor
	promedio de 7.54, calcule un intervalo de confianza con nivel del 99\% para la
	cantidad de manganeso verdadera que contiene ese mineral.
	
	\item \'{I}dem a), pero si se hicieron 10 mediciones.
	
	\item \textquestiondown Cu\'{a}ntas mediciones habr\'{\i}a que hacer para que
	el intervalo de confianza al 99\% tenga una longitud $\leq0.10$?
\end{enumerate}


\item \label{reduccion-peso} Se registr\'{o} el valor (en Kg) de la reducci\'{o}n del peso, de cada
uno de 16 pacientes elegidos al azar, despu\'{e}s de una semana de
tratamiento. El promedio de esos 16 valores fue de 3.42Kg. Suponga que la
p\'{e}rdida de peso luego de una semana de tratamiento es una variable
aleatoria con distribuci\'{o}n normal.

 Construya un intervalo de confianza del 99\% para el valor medio
poblacional de la reducci\'{o}n del peso despu\'{e}s de una semana de tratamiento, asumiendo  $\sigma=0.68$Kg. ($\sigma$ conocido). 


\item Considere nuevamente el problema planteado en el ejercicio \ref{reduccion-peso}. Construya un intervalo de confianza del 99\% para el valor medio
poblacional de la reducci\'{o}n del peso despu\'{e}s de una semana de tratamiento, pero ahora $\sigma$  es desconocido  y $s=0.68$Kg.
 Compare la longitud del  intervalo obtenido con el que contruyo en el ejercicio  \ref{reduccion-peso}. 


\item Diez mediciones de recuperaci\'{o}n de bromuro pot\'{a}sico por
cromatograf\'{\i}a de gas-l\'{\i}quido en muestras de tomates de cierta
partida arrojaron una media muestral de 782$\mu g/g$ y un desv\'{\i}o
$s=16.2\mu g/g$. Suponga que las mediciones tienen distribuci\'{o}n normal.

\begin{enumerate}
\item Halle un intervalo de confianza del 95\% para la media $\left(
\mu\right)  $ de las mediciones en esta partida de tomates.

\item Suponiendo que el error de medici\'{o}n debido al m\'{e}todo es
despreciable respecto a la variabilidad entre los tomates y que las 10
mediciones se realizan para cada uno de 10 tomates, \textquestiondown c\'{o}mo
debe interpretarse la media $\mu$? \textquestiondown C\'{o}mo debe
interpretarse $\sigma^{2}$?

\item En cambio, si las 10 mediciones se realizaron sobre el mismo tomate,
\textquestiondown c\'{o}mo puede interpretarse $\mu$?
\textquestiondown C\'{o}mo debe interpretarse $\sigma^{2}$?

\item Halle un intervalo de confianza del 95\% para la varianza $\left(
\sigma^{2}\right)  $ de las mediciones.

\item Halle un intervalo de confianza del 95\% para la desviaci\'{o}n
est\'{a}ndar $\left(  \sigma\right)  $ de las mediciones.
\end{enumerate}

\item Se hicieron varias mediciones del contenido de glucosa de una
soluci\'{o}n. Suponga que estas mediciones siguen un modelo de Gauss sin
sesgo, es decir, el modelo descripto en el ejercicio 2 de la Pr\'{a}ctica 3,
cuando los errores tienen distribuci\'{o}n normal.

\begin{enumerate}
\item Escriba el modelo. Se calcul\'{o} el intervalo de confianza del 95\%
para la media, que result\'{o} ser $(10.28,11.32)$. \textquestiondown Qu\'{e}
significa \textquotedblleft la media\textquotedblright\ en este problema?

\item Decir si es verdadero o falso, y explicar:

\begin{enumerate}
\item Un 95\% de las mediciones observadas pertenecen a ese intervalo.

\item Hay una probabilidad de 0.95 de que la pr\'{o}xima medici\'{o}n caiga en
el intervalo.

\item Alrededor del 95\% de las veces que uno realice el ensayo y construya el
intervalo de confianza, \'{e}ste contendr\'{a} la verdadera concentraci\'{o}n
de glucosa de la soluci\'{o}n.

\item La probabilidad de que el intervalo $(10.28,11.32)$ contenga a la
verdadera concentraci\'{o}n de glucosa es de 0.95.
\end{enumerate}
\end{enumerate}


\item Utilizando dos m\'{e}todos de an\'{a}lisis se hicieron determinaciones
del contenido de hierro de una muestra de un mineral. Se asume que las
determinaciones correspondientes a ambos m\'{e}todos tienen distribuci\'{o}n
normal cuyo desv\'io est\'andar es $\sigma=0,1 \%$. Los resultados obtenidos son los siguientes:%
\begin{align*}
\text{M\'{e}todo 1}\qquad n_{1}  &  =12\qquad\overline{x}=15.22\%\\
\text{M\'{e}todo 2}\qquad n_{2}  &  =11\qquad\overline{y}=15.30\%
\end{align*}
Calcule la estimaci\'on por intervalo de la diferencia de medias de ambos m\'etodos con nivel de significaci\'on del $95 \%$.

\item Este es un ejemplo en el que se desea estudiar si un cambio en las condiciones de un experimento afecta el resultado. Se est\'{a} estudiando un procedimiento para la determinaci\'{o}n de esta\~{n}o en productos alimenticios. Para ello se tomaron 12 muestras del mismo producto. Se
eligieron 6 de estas muestras al azar y se llevaron al punto de ebullici\'{o}n
con HCl a reflujo durante 30 minutos. Con las 6 muestras restantes el tiempo
de reflujo fue de 70 minutos. Los resultados fueron los siguientes:%
\[%
\begin{tabular}
[c]{cccc}%
Tiempo de reflujo (m)\qquad & Esta\~{n}o encontrado (mg/Kg) & $\overline{x}$ &
$s^{2}$\\
30 & 57, 57, 59, 56, 56, 59 & 57.33 & 1.867\\
70 & 57, 55, 58, 59, 59, 59 & 57.83 & 2.57
\end{tabular}
\ \
\]
Suponga que la cantidad de esta\~{n}o encontrada cuando el tiempo de reflujo
es de 30 minutos es una variable aleatoria con distribuci\'{o}n $N(\mu
_{1},\sigma_1^{2})$ y dicha distribuci\'{o}n es $N(\mu_{2},\sigma_2^{2})$
cuando el tiempo de reflujo es de 70 minutos. 
\begin{enumerate}
\item Calcule la estimaci\'on por intervalo del cociente de varianzas de esta\~{n}o para los dos tiempos de ebullici\'{o}n con nivel de significaci\'on del $95 \%$.
\item Suponga que $\sigma_1=\sigma_2$. Calcule el intervalo de confianza estimado con nivel de significaci\'on del $95 \%$ para la diferencia de contenido de esta\~{n}o para los dos tiempos de ebullici\'{o}n.
\end{enumerate}

\item Una f\'abrica de calzado desea comparar dos tipos de suela (marca A y marca B). Para ello, se coloca una  suela de 2 cm. en  cada zapato de 6 individuos y se miden las suelas de cada pie despu\'es de 3 meses. La tabla a continuaci\'on presenta el grosor de las suelas transcurrido ese tiempo.

\vspace{0.2cm}
	\begin{tabular}{|c|c|c|}\hline
		Individuo & Suela Izquierda (marca A)  &Suela Derecha (marca B)\\\hline\hline
1	&	1.43	&	1.42	\\
2	&	1.27	&	1.24	\\
3	&	1.48	&	1.39	\\
4	&	1.53	&	1.41	\\
5	&	1.71	&	1.6	\\
6	&	1.72	&	1.61	\\
\hline
\end{tabular}

Se sabe que la diferencia de grosores entre suelas de cada individuo se distribuye normalmente. Calcule la estimaci\'on por intervalo para la diferencia media de grosores entre marcas de suela a nivel 0,95 \textquestiondown Hay diferencia entre la calidad de las suelas? Justifique


\end{enumerate}

\end{document}
