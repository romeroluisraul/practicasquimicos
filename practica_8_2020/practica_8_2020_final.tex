\documentclass[11pt,a4paper,twoside]{article}%
\usepackage{amsmath}
\usepackage{amsfonts}
\usepackage{amssymb}
\usepackage{graphicx}%
\setcounter{MaxMatrixCols}{30}
%TCIDATA{OutputFilter=latex2.dll}
%TCIDATA{Version=5.50.0.2953}
%TCIDATA{CSTFile=40 LaTeX article.cst}
%TCIDATA{Created=Thursday, December 26, 2013 11:29:10}
%TCIDATA{LastRevised=Wednesday, March 05, 2014 16:56:34}
%TCIDATA{<META NAME="GraphicsSave" CONTENT="32">}
%TCIDATA{<META NAME="SaveForMode" CONTENT="1">}
%TCIDATA{BibliographyScheme=Manual}
%TCIDATA{<META NAME="DocumentShell" CONTENT="Standard LaTeX\Blank - Standard LaTeX Article">}
%BeginMSIPreambleData
\providecommand{\U}[1]{\protect\rule{.1in}{.1in}}
%EndMSIPreambleData
\newtheorem{theorem}{Theorem}
\newtheorem{acknowledgement}[theorem]{Acknowledgement}
\newtheorem{algorithm}[theorem]{Algorithm}
\newtheorem{axiom}[theorem]{Axiom}
\newtheorem{case}[theorem]{Case}
\newtheorem{claim}[theorem]{Claim}
\newtheorem{conclusion}[theorem]{Conclusion}
\newtheorem{condition}[theorem]{Condition}
\newtheorem{conjecture}[theorem]{Conjecture}
\newtheorem{corollary}[theorem]{Corollary}
\newtheorem{criterion}[theorem]{Criterion}
\newtheorem{definition}[theorem]{Definition}
\newtheorem{example}[theorem]{Example}
\newtheorem{exercise}[theorem]{Exercise}
\newtheorem{lemma}[theorem]{Lemma}
\newtheorem{notation}[theorem]{Notation}
\newtheorem{problem}[theorem]{Problem}
\newtheorem{proposition}[theorem]{Proposition}
\newtheorem{remark}[theorem]{Remark}
\newtheorem{solution}[theorem]{Solution}
\newtheorem{summary}[theorem]{Summary}
\newenvironment{proof}[1][Proof]{\noindent\textbf{#1.} }{\ \rule{0.5em}{0.5em}}
\topmargin -0.7in
\oddsidemargin -0.30in
\evensidemargin -0.7in
\textwidth 7.3in
\textheight 9.5in
\begin{document}


\begin{center}
\textbf{\textsf{Estad\'{\i}stica (Qu\'{\i}mica) - Primer Cuatrimestre 2020 - Coronavirus}}

\textbf{Pr\'{a}ctica 8 - Regresi\'{o}n Lineal\vspace{-0.1in}}

\end{center}

\begin{enumerate}
\item En una experiencia para calibrar un instrumento con el objeto de medir
la resistencia el\'{e}ctrica de cierto material, se obtuvieron las siguientes
mediciones:%
\[
\text{%
\begin{tabular}
[c]{ccccc}%
$X$ & \multicolumn{1}{|c}{60} & \multicolumn{1}{|c}{70} &
\multicolumn{1}{|c}{80} & \multicolumn{1}{|c}{100}\\\hline
$Y$ & \multicolumn{1}{|c}{38} & \multicolumn{1}{|c}{64} &
\multicolumn{1}{|c}{72} & \multicolumn{1}{|c}{110}\\
& \multicolumn{1}{|c}{44} & \multicolumn{1}{|c}{70} & \multicolumn{1}{|c}{76}
& \multicolumn{1}{|c}{118}\\
& \multicolumn{1}{|c}{50} & \multicolumn{1}{|c}{} & \multicolumn{1}{|c}{82} &
\multicolumn{1}{|c}{}%
\end{tabular}
}%
\]
donde $X$ es la resistencia el\'{e}ctrica (en ohms) determinada por un
m\'{e}todo suficientemente exacto como para ser considerado sin error e $Y$ es
la medici\'{o}n le\'{\i}da en el instrumento.

\begin{enumerate}
\item Encontrar un intervalo de confianza del 95\% para el valor esperado de
$Y$ cuando $X=90$ ohms.

\item Si se toma una nueva porci\'{o}n de material, se mide su resistencia
el\'{e}ctrica y \'{e}sta resulta 85 ohms, hallar un intervalo de
predicci\'{o}n del 95\% para la medici\'{o}n a\'{u}n no observada.

\item Suponiendo que se hace una nueva observaci\'{o}n independiente de las
anteriores y \'{e}sta resulta igual a 73, encontrar una regi\'{o}n de
confianza del 90\% para el verdadero valor de la resistencia el\'{e}ctrica que
corresponder\'{a} a ese material (este es el objeto fundamental de la
experiencia, pues se desea calibrar el instrumento de manera que luego se mida
s\'{o}lo $Y$ y con esto se pueda tener una idea de cu\'{a}l es la resistencia
el\'{e}ctrica $X$).
\end{enumerate}

\item Se analizaron 8 soluciones est\'{a}ndares de plata por
espectrometr\'{\i}a de absorci\'{o}n at\'{o}mica de llama. Se obtuvieron los
siguientes resultados:%
\[
\text{%
\begin{tabular}
[c]{|c|c|c|c|c|c|c|c|c|}\hline
Concentraci\'{o}n (ng/ml) & 0 & 10 & 20 & 30 & 40 & 50 & 60 & 70\\\hline
Absorbancia & 0.257 & 0.314 & 0.364 & 0.413 & 0.468 & 0.528 & 0.574 &
0.635\\\hline
\end{tabular}
}%
\]


\begin{enumerate}
\item Realice el gr\'{a}fico de calibraci\'{o}n, determine la pendiente y la
ordenada al origen de la recta de cuadrados m\'{\i}nimos.

\item Calcule el IC al 95\% para la pendiente. Lo mismo para la ordenada al origen.

\item Se han hecho 4 an\'{a}lisis para una nueva muestra. Los valores de
absorbancia observados son: 0.308, 0.314, 0.347 y 0.312. Estime la
concentraci\'{o}n de plata en esa muestra y calcule un IC al 95\% para dicha concentraci\'{o}n.
\end{enumerate}

\item Un investigador de marketing estudi\'{o} las ventas anuales de cierto
producto que fue introducido en el mercado hace 15 a\~{n}os. Los datos de la
siguiente tabla son%
\begin{align*}
X  & =\text{a\~{n}os transcurridos desde el lanzamiento al mercado del
producto}\\
Y  & =\text{ventas en miles de unidades.}%
\end{align*}%
\[
\text{%
\begin{tabular}
[c]{cc}%
$X$ & \multicolumn{1}{|c}{$Y$}\\\hline
1 & \multicolumn{1}{|c}{217}\\
2 & \multicolumn{1}{|c}{204}\\
3 & \multicolumn{1}{|c}{215}\\
4 & \multicolumn{1}{|c}{251}\\
5 & \multicolumn{1}{|c}{312}\\
6 & \multicolumn{1}{|c}{293}\\
7 & \multicolumn{1}{|c}{269}\\
8 & \multicolumn{1}{|c}{420}\\
9 & \multicolumn{1}{|c}{337}\\
10 & \multicolumn{1}{|c}{399}\\
11 & \multicolumn{1}{|c}{419}\\
12 & \multicolumn{1}{|c}{320}\\
13 & \multicolumn{1}{|c}{488}\\
14 & \multicolumn{1}{|c}{460}\\
15 & \multicolumn{1}{|c}{371}%
\end{tabular}
}%
\]


\begin{enumerate}
\item Represente estos datos en un diagrama de dispersi\'{o}n.
\textquestiondown Es razonable suponer que existe una relaci\'{o}n lineal
entre $X$ e $Y$ que permita predecir $Y$ en funci\'{o}n de $X$?

\item Realice un ajuste por cuadrados m\'{\i}nimos.

\item Obtenga un gr\'{a}fico de residuos versus valores ajustados.
\textquestiondown Le parece razonable el supuesto de igualdad de varianzas
para todas las observaciones?

\item Suponga que $Var(Y_{i})=kx_{i}$, defina los pesos adecuados y realice
una estimaci\'{o}n de la recta por m\'{\i}nimos cuadrados pesados.
\end{enumerate}

\item El archivo \texttt{aire.txt} contiene tres variables \texttt{ozono},
\texttt{temp} y \texttt{grupo}, correspondientes a el nivel de ozono y la
temperatura medidos en 108 d\'{\i}as elegidos al azar en una misma latitud,
longitud y altitud. La variable \texttt{grupo} es una variable categ\'{o}rica
que indica nivel de temperatura.

\begin{enumerate}
\item Suponga que s\'{o}lo dispone de la variable ozono, medida en la forma
indicada. \textquestiondown Puede asumirse que los datos de ozono corresponden
a una variable aleatoria con distribuci\'{o}n normal? Para responder a esta
pregunta haga un qqplot, y un test de normalidad con nivel 0.10. Halle un
intervalo de confianza de nivel 0.95 para la media poblacional del ozono.
Justifique la elecci\'{o}n del m\'{e}todo seguido para hallarlo. Estime la
varianza poblacional del ozono.

\item Sabiendo que las mediciones de ozono se realizaron en 4 tipos de
d\'{\i}as (clasificados en d\'{\i}as de temperatura baja, media, media alta y
alta) y que esta categorizaci\'{o}n se encuentra en la variable \texttt{grupo}%
, realice un ANOVA para ver si la categor\'{\i}a influye en el nivel medio de
ozono presente en un d\'{\i}a dado. Verifique los supuestos realizados,
concluya a nivel 0.05. Escriba el modelo ajustado, y d\'{e} un estimador de la
varianza de los datos. Compare este modelo con el modelo de errores
independientes del \'{\i}tem anterior.

\item Si ahora uno toma en cuenta, para modelar el comportamiento de la
variable \texttt{ozono}, que se cuenta con la temperatura medida en el mismo
momento que fue medida la variable ozono, (guardada en la variable
\texttt{temp}), proponga y ajuste un modelo de regresi\'{o}n lineal para estas
dos variables. Escriba el modelo, ajuste los par\'{a}metros, chequee la bondad
del ajuste. Escriba el modelo ajustado, y d\'{e} un estimador de la varianza
de los datos. Compare con el modelo anova anterior.
\end{enumerate}
\item Como parte de una investigaci\'{o}n se trata de establecer si se puede
utilizar la concentraci\'{o}n de estrona en saliva para predecir la
concentraci\'{o}n del esteroide en plasma libre. Se obtuvieron los siguientes
datos de 14 varones sanos, siendo $X:=$ concentraci\'{o}n de estrona en saliva
(en pg/ml) e $Y:=$ concentraci\'{o}n de estrona en plasma libre (en pg/ml).%
\[%
\begin{tabular}
[c]{ccccc}%
$X$ & \multicolumn{1}{|c}{$Y$} &  & $X$ & \multicolumn{1}{|c}{$Y$%
}\\\cline{1-2}\cline{4-5}%
7.4 & \multicolumn{1}{|c}{30.0} &  & 14.0 & \multicolumn{1}{|c}{49.0}\\
7.5 & \multicolumn{1}{|c}{25.0} &  & 14.5 & \multicolumn{1}{|c}{55.0}\\
8.5 & \multicolumn{1}{|c}{31.5} &  & 16.0 & \multicolumn{1}{|c}{48.5}\\
9.0 & \multicolumn{1}{|c}{27.5} &  & 17.0 & \multicolumn{1}{|c}{51.0}\\
9.0 & \multicolumn{1}{|c}{39.5} &  & 18.0 & \multicolumn{1}{|c}{64.5}\\
11.0 & \multicolumn{1}{|c}{38.0} &  & 20.0 & \multicolumn{1}{|c}{63.0}\\
13.0 & \multicolumn{1}{|c}{43.0} &  & 23.0 & \multicolumn{1}{|c}{68.0}%
\end{tabular}
\]


\begin{enumerate}
	\item Represente estos datos en un diagrama de dispersi\'{o}n.
	
	\item \textquestiondown Es razonable suponer que existe una relaci\'{o}n
	lineal entre $X$ e $Y$ que permita predecir $Y$ en funci\'{o}n de $X$?
	
	\item Calcule el coeficiente de determinaci\'{o}n $R^{2}$. Interprete el
	significado de esta medida en este ejemplo.
\end{enumerate}


\end{enumerate}


\end{document}
