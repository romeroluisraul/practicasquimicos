\documentclass[a4paper, 11pt]{article}
\usepackage{amsmath}
\usepackage[utf8]{inputenc}
\usepackage[spanish]{babel}
\usepackage{wrapfig, framed, caption}
\usepackage[usenames, dvipsnames]{color}
\usepackage{amssymb}
\usepackage[a4paper, total={7in, 9in}]{geometry}
\usepackage{graphicx}
\usepackage{float}
\usepackage{subcaption}
\usepackage{multicol}
\usepackage{xcolor}
\usepackage{hyperref}
\usepackage{verbatim}
\usepackage{enumitem}

%de proba
\def\P{{\mathbf P}}
\def\E{{\mathbf E}}
\def\cov{{\bf cov}}
\def\var{{\bf var}}

\newcommand{\PP}{\mathbb{P}}

% \setlist[enumerate]{font={\series}}
\setlist[enumerate,1]{label={\arabic*.}}
\setlist[enumerate,2]{label={(\alph*)}}
\setlist[enumerate,3]{label={\roman*.}}

\begin{document}


\begin{center}
\textbf{\textsf{Estad\'{\i}stica (Qu\'{\i}mica) }}\\
\vspace{0.2cm}
\textbf{Pr\'{a}ctica 1 - Probabilidad\vspace{-0.1in}}
\end{center}


\begin{enumerate}

\item \label{propiedades} Sea $(\mathbb{P}, \Omega)$ un espacio de probabilidad. Demuestre que:\vspace{-0.2cm}

\begin{enumerate}
    \item $\mathbb{P}(\emptyset) = 0$
    \item \label{suma} Si $A_1, \ldots, A_n \in \Omega$ y son eventos disjuntos, entonces $\mathbb{P}\left(\bigcup_{i=1}^n A_i\right) = \sum_{i=1}^n \mathbb{P}(A_i)$. Observe que esto prueba en particular que $\mathbb{P}(A_1 \cup A_2) = \mathbb{P}(A_1) + \PP(A_2)$.
    \item Si $A \in \Omega$ entoces $\mathbb{P}(A^c) = 1-\PP(A)$
\end{enumerate}


% \item (Opcional) Definamos $P:\mathcal{P}(\mathbb{N})\to[0,1]$ dado por,
% \begin{equation*}
%     P(A) = \sum_{n \in A} \frac{1}{2^n}.
% \end{equation*}
% Demuestre que $(P,\mathcal{P}(\mathbb{N}))$ es un espacio de probabilidad.

\item Definamos $P:\mathcal{P}(\{0,1,2,3,4\})\to[0,1]$ dado por,
\begin{equation*}
    P(A) = \frac{\# A}{5}.
\end{equation*}
Demuestre que $(P,\mathcal{P}(\{0,1,2,3,4\}))$ es un espacio de probabilidad.


\item Se arroja 3 veces una moneda equilibrada y se observa la secuencia de
caras y cecas.\vspace{-0.2cm}


\begin{enumerate}
\item Describa el espacio muestral. \textquestiondown Es equiprobable? Asigne probabilidad a cada uno de los  posibles resultados.


\item Calcule las probabilidades de los siguientes sucesos:\vspace{-0.1cm}


\begin{enumerate}
\item A : salieron al menos dos caras


\item B : en los dos primeros tiros salieron caras


\item C : en el \'{u}ltimo tiro sali\'{o} ceca


\item no ocurri\'{o} el suceso A


\item ocurrieron los sucesos A y B simult\'{a}neamente


\item ocurri\'{o} alguno de los dos sucesos A \'{o} B.
\end{enumerate}
\end{enumerate}




\item Se lanzan dos dados equilibrados de 6 caras y se anotan los n\'umeros de las caras observadas.
\begin{enumerate}
\item Describa un espacio muestral de este experimento que sea equiprobable.
\item Calcule la probabilidad de que se anoten los n\'umeros 2 y 5.
\item Calcule la probabilidad de que se hayan observado dos 1.
\item Calcule la probabilidad de que la suma de los n\'umeros observados sea 6.
\end{enumerate}




\item \label{bolitas} Una caja contiene 3 bolitas rojas, 2 azules y 4 blancas. Se extraen 2
bolitas \textbf{con reposici\'{o}n}. Calcule la probabilidad de
obtener:\vspace{-0.1cm}


\begin{enumerate}
\item las dos bolitas del mismo color.
\item al menos una bolita roja.
\item una bolita azul y una roja.
\item una bolita azul o una roja.
\end{enumerate}


\item Rehaga el ejercicio \ref{bolitas} pero con las extracciones realizadas \textbf{sin
reposici\'{o}n}.





\item Se arroja un dado seis veces. \textquestiondown Cu\'{a}l de las
siguientes opciones ofrece la mejor manera de ganar? \textquestiondown O son
equivalentes?\vspace{-0.15cm}


\begin{enumerate}
\item Ganar \$1 si sale al menos un as.


\item Ganar \$1 si sale un as todas las veces.


\item Ganar \$1 si sale la secuencia 1, 2, 3, 4, 5, 6 (en ese orden).


\item Ganar \$1 si los dos primeros n\'{u}meros que salen son iguales.
\end{enumerate}





\item Se realiza una experiencia que consiste en provocar una reacci\'{o}n y
luego  se registra  (i) el nivel (bajo, medio, alto) de presi\'{o}n al finalizar la
reacci\'{o}n y (ii) si la reacci\'{o}n se completa antes de
los 10 minutos o pasados los 10 minutos. Las probabilidades que dicha
reacci\'{o}n se complete antes de los 10 minutos y con distintos niveles de
presi\'{o}n en un d\'{\i}a cualquiera son conocidas y se muestran en la primer fila de 
siguiente tabla.\vspace{-0.1cm}%
\[%
\begin{tabular}
[c]{cc|c|c|c|}\cline{3-5}
&  & \multicolumn{3}{|c|}{Niveles de presi\'{o}n}\\\cline{3-5}
&  & bajo & medio & alto\\\hline
\multicolumn{1}{|c|}{Tiempo } & \multicolumn{1}{|c|}{%
	%TCIMACRO{\TEXTsymbol{<} }%
	%BeginExpansion
	$<$
	%EndExpansion
	10 minutos} & $0.05$ & $0.15$ & $0.40$\\\cline{2-5}%
\multicolumn{1}{|c|}{de reacci\'{o}n} & \multicolumn{1}{|c|}{$\geq$ 10
	minutos} & $0.10$ &  & $0.20$\\\hline
\end{tabular}
\]




\begin{enumerate}
	\item Complete el cuadro. \textquestiondown Cu\'{a}l es la probabilidad de que
	la reacci\'{o}n se complete antes de los 10 minutos y con un nivel de
	presi\'{o}n alto?
	
	
	\item \textquestiondown Cu\'{a}l es la probabilidad de que la reacci\'{o}n no
	se produzca a nivel de presi\'{o}n alto y se produzca antes de los diez minutos?
	
	\item \textquestiondown Cu\'{a}l es la probabilidad de que el tiempo de
	reacci\'{o}n sea menor a los 10 minutos?
	


	\item \textquestiondown A qu\'{e} nivel de presi\'{o}n es m\'{a}s probable que
	se produzca la reacci\'{o}n?
	
%	\item \textquestiondown Son m\'{a}s probables las reacciones que se producen
%	antes de los 10 minutos?
%	
\end{enumerate}


%\newpage


\item Consideremos nuevamente las condiciones del ejercicio anterior.
\begin{enumerate} 
	\item Sabiendo que  una reacci\'{o}n
se produjo antes de los 10 minutos,
\textquestiondown cu\'{a}l es la probabilidad de que  haya sido a nivel de presi\'{o}n media?




	\item Sabiendo que el nivel de presi\'{o}n fue alto,
	\textquestiondown cu\'{a}l es la probabilidad de que la reacci\'{o}n haya
	tenido lugar antes de los 10 minutos?
	\item Considere los siguientes sucesos\vspace{-0.1cm}
	\begin{align*}
	A  & :\text{el tiempo de reacci\'{o}n fue menor a 10 minutos}\\
	B  & :\text{el nivel de presi\'{o}n fue bajo}%
	\end{align*}
	Calcule $P(A),P(B),P(A\mid B),P(B\mid A)$, $P\left(  B^c \mid A\right)
	,P(B\mid A^{c})$ y $P(A\cap B)$.
	
	\item \textquestiondown Es el nivel de presi\'{o}n independiente del tiempo de
	reacci\'{o}n? Justifique su respuesta.
	
\end{enumerate}


\item Sean $A,B \in \Omega$. Demueste que las siguientes afirmaciones son equivalentes:
\begin{enumerate}
    \item $A$ y $B$ son independientes.
    \item $A^c$ y $B$ son independientes.
    \item $A$ y $B^c$ son independientes.
    \item $A^c$ y $B^c$ son independientes.
\end{enumerate}








\item Se realiza el mismo experimento del ejercicio anterior pero se registra apenas el nivel de presi\'on de la reacci\'{o}n. Recordemos que la probabilidad de cada uno de los niveles de
presi\'{o}n al completarse la reacci\'{o}n son:\vspace{-0.1cm}%
\[
P(\text{bajo})=0.15\qquad P(\text{medio})=0.25\qquad P(\text{alto})=0.60
\]
Se repite la experiencia en 2 d\'{\i}as sucesivos en condiciones
independientes e id\'{e}nticas.\vspace{-0.2cm}


\item \textquestiondown Cu\'antos invitados necesit\'as en una fiesta para que al menos dos cumplan a\~nos el mismo d\'ia? \textquestiondown Cu\'antos para que al menos dos cumplan a\~nos el mismo d\'ia con probabilidad mayor o igual a 0,5? \textquestiondown Y mayor o igual a 0,95?\footnote{Cliqueando \hyperlink{https://www.lanacion.com.ar/sociedad/desafio-que-es-paradoja-del-cumpleanos-nid2340525}{ac\'a} encontrar\'as una nota publicada en un diario sobre este problema.} Asuma que un a\~no tiene 365 d\'ias e implemente, en R, una funci\'on que tenga como entrada la cantidad de invitados y devuelva, \verb|cumple(invitados)|, la probabilidad de que al menos dos cumplan a\~nos el mismo d\'ia. Grafique en el eje $x$ el n\'umero de invitados y en el eje $y$ la probabilidad calculada para cada n\'umero de invitados.




\begin{enumerate}
	\item Describa el espacio muestral asociado a este experimento y calcule la probabilidad de cada uno de los posibles resultados.
	
	\item Calcule  la probabilidad de cada uno de los siguientes eventos:
	
	\begin{enumerate}
		\item  la reacci\'{o}n se completa con un nivel de presi\'{o}n bajo el primer d\'{\i}a.
		
		\item  la reacci\'{o}n se completa con un nivel de presi\'{o}n bajo en los dos d\'{\i}as.
		
		\item  la reacci\'{o}n se completa por lo menos en un d\'{\i}a con un nivel de presi\'{o}n bajo.
		
		\item  la reacci\'{o}n se completa a lo sumo un d\'{\i}a con nivel de presi\'{o}n alto.
	\end{enumerate}
	
%	\item Suponga ahora que \'{u}nicamente interesa si el nivel de presi\'{o}n es
%	bajo o si no es bajo. Rehaga los \'{\i}tems a) y b) i) ii) iii).
\end{enumerate}




\item Rehaga el ejercicio 1 suponiendo que la moneda est\'{a} cargada de
manera tal que la probabilidad de obtener cara es 3/4. %independencia


\item En una materia que se dict\'{o} el primer cuatrimestre del a\~{n}o
pasado, la distribuci\'{o}n de la frecuencia de notas obtenidas fue la
siguiente:\vspace{-0.1cm}%
\[%
\begin{tabular}
[c]{c|c}%
Nota & Porcentaje\\\hline
0 y 1 & 5\%\\
2 y 3 & 15\%\\
4 a 7 & 50\%\\
8 a 10 & 30\%
\end{tabular}
\]
\vspace{-0.2cm}


La materia se aprueba con una nota mayor o igual a 4.


\begin{enumerate}
\item \vspace{-0.1cm}Se elige un estudiante al azar. Hallar la probabilidad de
que haya aprobado.


\item Sabiendo que hubo 200 alumnos en el curso, si se eligen sin
reposici\'on dos estudiantes al azar, \textquestiondown cu\'{a}l es la probabilidad de que ambos
hayan aprobado? \textquestiondown Cu\'{a}l es la probabilidad de que al menos
uno haya aprobado?


\item Responder b) si a los estudiantes se los elige con reposici\'on. Compare las probabilidades halladas en este \'item con las del b), \textquestiondown son muy diferentes entre s\'{\i}?


\item Felipe curs\'{o} dicha materia el cuatrimestre pasado y la aprob\'{o}.
\textquestiondown Cu\'{a}l es la probabilidad de que haya sacado m\'{a}s de 7?


\end{enumerate}


\item Una caja contiene tres monedas: una de ellas es de curso legal (equilibrada), otra tiene dos caras y la restante est\'a cargada de modo que la probabilidad de obtener cara es $\frac{1}{5}$. Se selecciona una moneda al azar y se lanza.
\begin{enumerate}
\item Hallar la probabilidad de que salga cara.
\item Sabiendo que sali\'o cara, hallar la probabilidad de que se haya extra\'ido la moneda que hace m\'as probable que salga ceca.
\end{enumerate}



\end{enumerate}
\end{document}
