\documentclass[a4paper, 11pt]{article}
\usepackage{amsmath}
\usepackage[utf8]{inputenc}
\usepackage[spanish]{babel}
\usepackage{wrapfig, framed, caption}
\usepackage[usenames, dvipsnames]{color}
\usepackage{amssymb}
\usepackage[a4paper, total={7in, 9in}]{geometry}
\usepackage{graphicx}
\usepackage{float}
\usepackage{subcaption}
\usepackage{multicol}
\usepackage{xcolor}
\usepackage{hyperref}
\usepackage{verbatim}
\usepackage{enumitem}

%de proba
\def\P{{\mathbf P}}
\def\E{{\mathbf E}}
\def\cov{{\bf cov}}
\def\var{{\bf var}}




\begin{document}


\begin{center}
\centerline{\bf Estadística 1 (Química)}%
\vspace{0.2cm}
\textbf{Práctica 3 - Sumas de variables aleatorias\vspace{-0.1in}}
\end{center}


\begin{enumerate}
\item Se realizan mediciones independientes del volumen inicial $\left(
X\right)  $ y final $\left(  Y\right)  $ en una bureta. Supongamos que las
mediciones inicial y final siguen el modelo de errores independientes, es
decir,
\[
X=\alpha+\varepsilon_{X},\qquad Y=\beta+\varepsilon_{Y}%
\]
donde $\alpha$ y $\beta$ son los volúmenes desconocidos, $\varepsilon_{X}$
y $\varepsilon_{Y}$, los errores de medición, son variables aleatorias
independientes con media 0 y varianza $\sigma^{2}$. Sea $Z=Y-X$.


\begin{enumerate}
\item Hallar $\mathbb{E}(Z)$ y $\mathbb{V}(Z)$.


\item En una titulación, la lectura inicial en una bureta es de 0.00ml y
la lectura final es de 15.60ml. Para ambas mediciones se sabe que el error de
medición tiene una desviación estándar de 0.05ml.


\begin{enumerate}
\item Calcular el valor estimado del volumen utilizado.


\item \textquestiondown Cuál es la desviación estándar de su error
de medición?
\end{enumerate}
\end{enumerate}








\item \label{medidas repetidas} Modelo para medici\'on  con error aditivo.  Se 
desea determinar una magnitud $\mu.$ Para ello se realizar\'an $n$ medidas 
repetidas, es decir, se realizarán $n$ mediciones de la misma magnitud en 
idénticas condiciones, que denotaremos con $X_{1},\dots,X_{n}$. Asumimos 
el siguiente modelo para las variables aleatorias $X_{i}$
\[
X_{i}=\mu+\varepsilon_{i}
\]
donde $\mu$ es la verdadera magnitud desconocida, y $\varepsilon_{i}$ es la
variable aleatoria que denota el error de la iésima medición. Asumimos
que $\varepsilon_{1},\dots,\varepsilon_{n}$ son variables aleatorias
independientes e idénticamente distribuidas (v.a.i.i.d.) con esperanza
cero y varianza $\sigma^{2}=0.25$. Notar que los errores son no observables. El
supuesto de que $\mathbb{E}(\varepsilon_{i})=0$ refleja la creencia en que el
método de medición empleado es exacto. Es decir que no produce errores
sistemáticos. La varianza del error, 
$\sigma^{2}=\mathbb{V}(\varepsilon_{i})$
representa la precisión del método de medición empleado. Sea
\[
\overline{X}_{n}=\frac{1}{n}\sum_{i=1}^{n}X_{i}
\]
el promedio (o media muestral) de las $n$ observaciones. 




Asumir ahora que el error de medici\'on tiene una distribuci\'on normal con media cero. Este modelo
probabil\'istico se conoce como el Modelo de Gauss sin sesgo. Asumir tambi\'en que el error de medición tiene desv\'io est\'andar $\sigma=0.5$, o sea que $\varepsilon_{i}\sim\mathcal{N}(0;0.25)$.
\begin{enumerate}
	\item  Obtener la distribución de $\overline{X}_{n}$, su esperanza y
	su varianza.
	
	\item Calcular la probabilidad de que el promedio de $n=10$ mediciones y
	de $n=100$ mediciones diste de la verdadera magnitud $\mu$ en menos de 0.1
	unidades. Notar que no fue necesario conocer el valor de $\mu$ para realizar
	este cálculo. 
	
	\item Obtener una expresión para la probabilidad de que el promedio de 
$n$ mediciones diste de la verdadera magnitud $\mu$ en menos de $0.1$ 
unidades en función de $n$. Estudiar la  monotonía y el límite 
cuando $n$ tiende a infinito de esta probabilidad.
	
	\item  Determinar cu\'an grande debe ser $n$ para que $P\left(  |\overline
	{X}_{n}-\mu|<0.1\right)  \geq0.99$. 
	
	
{\color{red}\item Para incluir en el futuro, en alguna parte de esta practica. Calcular la siguiente probabilidad
$$\mathbb P\left(\overline X_n-\frac{1.96*\sqrt{0.25}}{\sqrt{n}} <\mu <\overline X_n+\frac{1.96*\sqrt{0.25}}{\sqrt{n}}\right)$$}	
\end{enumerate}


\item \label{medidas repetidas_bis} Considerar nuevamente el modelo de 
mediciones propuesto en el ejercicio anterior, suponiendo ahora que 
$\varepsilon_i$  tiene una distribución \textbf{desconocida}, pero se sabe 
que 
$$E(\varepsilon_i)=0\;, \quad V(\varepsilon_i)=0.25$$




\begin{enumerate}
	\item Hallar $\mathbb{E}(\overline{X}_{n})$ y 
$\mathbb{V}(\overline{X}_{n})$.
	
	\item Para $n=10$ y $n=100$ mediciones, usando la desigualdad de Chebyshev,
	encontrar una cota inferior para
	\[
	P\left(  |\overline{X}_{n}-\mu|<0.1\right)  .
	\]
	Comparar el resultado obtenido con el hallado en el ítem (b) del 
ejercicio anterior.
	
	\item Determinar cu\'an grande debe ser $n$ para que $P\left(  |\overline{X}%
	_{n}-\mu|<0.1\right)  \geq0.99$, usando nuevamente la desigualdad de Chebyshev.
		Comparar el resultado obtenido con el valor hallado en el ítem 
(d) del ejercicio anterior.
	
\end{enumerate}




\item Se desea conocer la proporción de personas que están a favor de
la despenalización del aborto en una ciudad. Sea $p$ la proporción
poblacional que está a favor de la despenalización. Observar que $p$ es
un número fijo y desconocido. Para estimar a $p$ se eligen $n$ personas al
azar y se les pregunta a cada una de ellas su opinión. Para $i$ entre $1$
y $n$, sean%
\[
X_{i}=\left\{
\begin{array}
[c]{lll}%
1 &  & \text{si la iésima persona encuestada está a favor de la
despenalización}\\
&  & \\
0 &  & \text{en otro caso.}%
\end{array}
\right.
\]
Asumimos que las $X_{i}$ son v.a.i.i.d.


\begin{enumerate}
\item Expresar la proporción muestral de encuestados a favor de la
despenalización en términos de las variables $X_{i}.$


\item Proponga una expresión que sea una variable aletoria que permita aproximar $p$ para $n$ suficientemente grande. En adelante denominaremos a esta variable aleatoria \textbf{estimador}.


\item Hallar una cota superior para la probabilidad de que el estimador y el
verdadero parámetro $p$ difieran en más de 0.1, que no dependa de
valores desconocidos. Es decir, acotar superiormente la siguiente
expresión,
\[
P\left(  \left\vert \overline{X}_{n}-p\right\vert >0.1\right)
\]
de modo que la cota sólo dependa de $n$. \textquestiondown Qué pasa
con esta probabilidad cuando $n$ aumenta? \textquestiondown Cómo puede el
encuestador mejorar su estimación de $p$?
\end{enumerate}
%
%\item \label{medidas repetidas}Se desea determinar una magnitud $\mu.$ Para
%ello se realizan $n$ medidas repetidas, es decir, se realizarán $n$
%mediciones de la misma magnitud en idénticas condiciones, que denotaremos
%con $X_{1},...,X_{n}$. Asumimos el siguiente modelo para las variables
%aleatorias $X_{i}$%
%\[
%X_{i}=\mu+\varepsilon_{i}%
%\]
%donde $\mu$ es la verdadera magnitud desconocida, y $\varepsilon_{i}$ es la
%variable aleatoria que denota el error de la iésima medición. Asumimos
%que $\varepsilon_{1},...,\varepsilon_{n}$ son variables aleatorias
%independientes e idénticamente distribuidas (v.a.i.i.d.) con esperanza
%cero y varianza $\sigma^{2}=0.25$. Note que los errores son no observables. El
%supuesto de que $E(\varepsilon_{i})=0$ refleja la creencia en que el
%método de medición empleado es exacto. Es decir no produce errores
%sistemáticos. La varianza del error, $\sigma^{2}=Var(\varepsilon_{i})$
%representa la precisión del método de medición empleado. Sea
%\[
%\overline{X}_{n}=\frac{1}{n}\sum_{i=1}^{n}X_{i}%
%\]
%el promedio (o media muestral) de las $n$ observaciones.
%
%\begin{enumerate}
%\item Halle $E(\overline{X}_{n})$ y $Var(\overline{X}_{n})$.
%
%\item Para $n=10$ y $n=100$ mediciones, usando la desigualdad de Chebyshev,
%encuentre una cota inferior para
%\[
%P\left(  |\overline{X}_{n}-\mu|<0.1\right)  .
%\]
%
%
%\item Determine cuan grande debe ser $n$ para que $P\left(  |\overline{X}%
%_{n}-\mu|<0.1\right)  \geq0.99$, usando nuevamente la desigualdad de Chebyshev.
%\end{enumerate}
%
%Asuma ahora que el error de medición tiene una distribución normal con
%media cero. Este modelo probabilístico se conoce como el Modelo de Gauss
%sin sesgo. Asuma también que la desviación estándar es $0.5$, es
%decir $\varepsilon_{i}\sim\mathcal{N}(0,0.25)$.
%
%\begin{enumerate}
%\item[(d)] Obtenga la distribución de $\overline{X}_{n}$, su esperanza y
%su varianza.
%
%\item[(e)] Calcule la probabilidad de que el promedio de $n=10$ mediciones y
%de $n=100$ mediciones diste de la verdadera magnitud $\mu$ en menos de 0.1
%unidades. Note que no fue necesario conocer el valor de $\mu$ para realizar
%este cálculo. Compare los resultados obtenido con los valores hallados en
%el ítem b).
%
%\item[(f)] Obtenga una expresión para la probabilidad de que la
%medición diste de la verdadera magnitud $\mu$ en menos de $0.1$ unidades
%en función de $n$. Estudie su monotonía y el límite cuando $n$
%tiende a infinito de esta probabilidad.
%
%\item[(g)] Determine cuan grande debe ser $n$ para que $P\left(  |\overline
%{X}_{n}-\mu|<0.1\right)  \geq0.99$. Compare el resultado obtenido con el valor
%hallado en el ítem c).
%\end{enumerate}


\item Considerar nuevamente el modelo propuesto en el ejercicio
\ref{medidas repetidas}, asumiendo ahora que los errores se 
distribuyen de manera uniforme: $\varepsilon_{i}\sim
\mathcal{U}(-\sqrt{3}/2,\sqrt{3}/2)$. Calcular de forma aproximada la
probabilidad $P\left(  |\overline{X}_{n}-\mu|<0.1\right)  $, para $n=100$ y
determinar cu\'an grande debe ser $n$ para que $P\left(  |\overline{X}_{n}%
-\mu|<0.1\right)  \geq0.99$. Comparar con los resultados obtenidos en  el 
ejercicio \ref{medidas repetidas} y en el ejercicio \ref{medidas repetidas_bis}. 
\textquestiondown Qu\'e observa?


%\item La figura siguiente muestra
%\[%
%%TCIMACRO{\FRAME{itbpFX}{15.6531cm}{8.1693cm}{0cm}{}{}%
%%{graficobinomialpracticacontinuas.jpg}{\special{ language "Scientific Word";
%%type "GRAPHIC";  maintain-aspect-ratio TRUE;  display "USEDEF";
%%valid_file "F";  width 15.6531cm;  height 8.1693cm;  depth 0cm;
%%original-width 7.6666in;  original-height 3.9894in;  cropleft "0";
%%croptop "0.9989";  cropright "1.0003";  cropbottom "0";
%%filename 'graficobinomialpracticacontinuas.JPG';file-properties "XNPEU";}}}%
%%BeginExpansion
%\raisebox{-0cm}{\fbox{\includegraphics[
%trim=0.000000in 0.000000in -0.002300in 0.004388in,
%natheight=3.989400in,
%natwidth=7.666600in,
%height=8.1693cm,
%width=15.6531cm
%]%
%{graficobinomialpracticacontinuas.jpg}%
%}}%
%%EndExpansion
%\]
%la función de probabilidad puntual en $k$ graficada sobre el intervalo de
%longitud uno centrado en $k,$ para todos los enteros $k$ entre 0 y 100, junto
%con una curva normal $\mathcal{N}(\mu,\sigma^{2}).$ La función recién
%descripta está definida por
%\[
%g\left(  x\right)  =\left\{
%\begin{array}
%[c]{lll}%
%p_{X}\left(  k\right)  &  & \text{si }k-\frac{1}{2}\leq x<k+\frac{1}{2},\text{
%para }k\text{ entero entre 0 y 100}\\
%&  & \\
%0 &  & \text{en otro caso}%
%\end{array}
%\right.
%\]
%donde $X\sim Bi\left(  100,0.5\right)  .$\allowbreak
%
%\begin{enumerate}
%\item \textquestiondown Cuáles son los valores de $\mu$ y $\sigma^{2}$ que
%corresponden a la curva normal dibujada?
%
%\item La probabilidad de obtener 52 caras en 100 tiradas de una moneda
%equilibrada, \textquestiondown es exactamente igual al área entre $51.5$ y
%$52.5$ bajo la curva normal o al área bajo el gráfico de $g$?
%\textquestiondown Son muy distintas entre sí? \textquestiondown Qué
%resultado teórico está usando?
%
%\item La probabilidad de obtener 52 caras o más en 100 tiradas de una
%moneda equilibrada puede ser aproximada por \textquestiondown cuál zona
%del gráfico? Calcúlela.
%\end{enumerate}


\item Se tira 100 veces un dado de 6 caras. Usar la aproximación normal
para hallar la probabilidad de que:


\begin{enumerate}
\item salga \textquotedblleft6\textquotedblright\ entre 15 y 20 veces, inclusive.


\item la suma de los resultados obtenidos sea menor que 300.


\item el número de veces que el resultado sea par esté entre 40 y 60
veces, inclusive.


\item el número de veces que el resultado sea par sea mayor o igual que el
número de veces que el resultado sea impar.
\end{enumerate}


\item El cromato y dicromato de postasio forman soluciones intensamente coloreadas. Los iones de cromato y dicromato se encuentran en equilibrio químico dependiente del pH. Se quiere estudiar si el dióxido de carbono burbujeado puede alterar el equilibrio de una solución de cromato y desplazarla al dicromato. Un sensor mide la absorbancia característica del cromato, cuando la reacción se desplaza al dicromato, se activa un mecanismo que automáticamente agrega hidróxido de potasio hasta revertir el efecto del dióxido de carbono. El tiempo (en minutos) que la solución demora en desplazarse al dicromato es una variable aleatoria Exponencial con parámetro
$\lambda=2$. Se puede suponer que los tiempos que tarda la solución en desplzarse completamente son independientes entre sí.


\begin{enumerate}
\item Un investigador se va 30 minutos de su puesto y deja programadas 62
alícuotas de hidróxido de potasio
\textquestiondown Cuál es, aproximadamente, la probabilidad de que el investigador encuentre la solución con la coloración del cromato?


\item \textquestiondown Cuántas alícuotas de hidróxido de potasio tiene que dejar si quiere encontrar la solución sin desplazarse a dicromato con una probabilidad aproximada
mayor o igual que $0.99$?
\end{enumerate}


%\item Comparar lo obtenido en el ejercicio anterior con los resultados que se obtienen sin aproximar utilizando R. \textquestiondown Es buena la aproximaci\'on?


\item Para rellenar una zona del río se utilizan 2 camiones (A y B). La
distribución de la carga diaria (en toneladas métricas) transportada
por el camión A tiene función de densidad
\[
f\left(  x\right)  =\left\{
\begin{array}
[c]{lll}%
\frac{x}{52} &  & \text{si }11<x<15,\\
&  & \\
0 &  & \text{en otro caso}%
\end{array}
\right.
\]
El camión B lleva una carga diaria en toneladas con esperanza $18$ Tm y
desvío estándar $1.3$ Tm.


\begin{enumerate}
\item Calcular esperanza y varianza de la carga diaria transportada por A.


\item Calcular esperanza y desvío estándar de la carga total llevada
por los dos camiones en un día, asumiendo que las cargas transportadas
por ambos camiones son independientes.


\item Calcular \textbf{aproximadamente} la probabilidad de que la carga total
transportada en 256 días esté entre 7950 y 8000 Tm.  Asuma que las
cargas transportadas en días distintos son independientes.


\item \textquestiondown Puede calcular la probabilidad pedida en c)
\textbf{exactamente}?
\end{enumerate}


\item Supongamos que el peso (en gramos) de un frasco de mermelada sigue una
distribución normal con un peso medio de $\mu_{A}$ para la marca A y
$\mu_{B}$ para la marca B, y con desvíos $\sigma_{A}=8g$ y $\sigma
_{B}=6g$, respectivamente. Se eligen al azar $n_{A}=40$ frascos de la marca A
y $n_{B}=35$ frascos de la marca B, y se pesan sus contenidos.


\begin{enumerate}
\item \label{9a} Defina las variables aleatorias involucradas en este problema. Indique
la distribución exacta o aproximada de las medias muestrales asociadas a
ambas marcas. Indique la distribución exacta o aproximada de la diferencia
entre la media muestral para la marca A y la media muestral para la marca B.
Enuncie los resultados teóricos en los que se basa. Observe que no es
necesario conocer los valores de $\mu_{A}$\ y $\mu_{B}$\ sino sólo de la
diferencia para poder determinar la distribución de $\overline{X}_{n_{A}%
}-\overline{Y}_{n_{B}}.$


\item Asumamos que los pesos medios o esperados de ambas marcas son iguales.
Calcular la probabilidad de que la distancia entre las medias muestrales sea a
lo sumo 3. 


\item Calcular la probabilidad de que la distancia anterior sea por lo menos 5,
asumiendo que los pesos medios de ambas marcas son iguales.


\item No se conocen los valores de $\mu_{A}$\ y $\mu_{B}$, sólo se sabe
que el valor observado de la diferencia entre la \textbf{media muestral} para
la marca A y la \textbf{media muestral} para la marca B es 5,
\textquestiondown sería razonable inferir que $\mu_{A}-\mu_{B}=0$? Para
contestar calcule, asumiendo que $\mu_{A}-\mu_{B}=0$, la probabilidad de que
la distancia entre las medias muestrales sea 5 ó más. Si esta
probabilidad fuera \textquotedblleft muy peque\~{n}a\textquotedblright\ uno
tendería a dudar de que la diferencia entre las medias poblacionales es 0.


\item \textquestiondown Se modifican las distribuciones de \ref{9a}) si $n_{A}=400$ y
$n_{B}=350$? Justificar.
\end{enumerate}
\end{enumerate}

\end{document}
